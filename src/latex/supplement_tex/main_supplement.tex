\documentclass[a4paper,12pt]{article}

%%%
%%% Settings
%%%
\usepackage{amsmath}
\usepackage[english]{babel}
\usepackage{graphicx}
\usepackage{geometry}
\usepackage[comma,super]{natbib}
\bibliographystyle{ajhg}
\usepackage[section]{placeins}
\geometry{left=20mm, top=20mm}
\newcommand{\matr}[1]{\mathbf{#1}}

% rename figures to supplemental figures
\addto\captionsenglish{\renewcommand{\figurename}{\textbf{Figure}}}
\renewcommand{\thefigure}{\textbf{S\arabic{figure}}}

% remove page numbers
\pagenumbering{gobble}
\graphicspath{{../../../Output/paper_figures/},{../../Output/paper_figures/},{../Output/paper_figures/},{Output/paper_figures/}}

%%%
%%% MAIN DOCUMENT
%%%
\begin{document}


\section{Supplementary Figures}

%% Figure S1
\begin{figure}[h]
    \centering
    \includegraphics[width=\textwidth]{FigureS1_filtering.png}
    \caption{
        \textbf{Filtering of introns.}
        \textbf{(a)} Histogram of the raw intron coverage per sample-intron pair
for the Kremer data set grouped according to the intron filter status. Green
indicates that the intron passed the filter and orange indicates that the
intron was filtered out. \textbf{(b-d)} Same as (a), but for different tissues
in the GTEx data set. \textbf{(e)} Barplots of the number of introns passed
the filtering, splice sites passed the filtering, observed introns, and
samples per tissue within the GTEx data set used in this study.}
\end{figure}
\pagebreak


%% Figure S2
\begin{figure}[h]
	\centering
 	\includegraphics[width=\textwidth]{FigureS2_heatmap_psi3.png}
 	\caption{
 	    \textbf{Tissue-specific correlation structure for $\psi_3$.}
 	    \textbf{(a-c)} Intron-centered and logit-transformed $\psi_3$ of the 
10,000 most variable introns clustered by samples (columns and rows)
for three representative GTEx tissues: (a) suprapubic skin, (b)
left ventricle heart, (c) and whole blood. Red and blue depict relative high and 
low intron usage, respectively. Colored horizontal tracks display sequencing
center, batch, RNA integrity number (RIN), gender, age, and cause of death 
(DTHHRDY, Hardy scale classification) of the samples. \textbf{(d)} Boxplot of
absolute values of between-sample correlations of row-centered logit-transformed
$\psi_3$ for 48 GTEx tissues before (orange) and after (green) correction 
for the latent space. The intron-centered $\psi_3$ values were clipped to the 
$[0.01, 0.99]$ interval before logit-transformation.
}
\end{figure}
\pagebreak

%% Figure S3
\begin{figure}[h]
	\centering
	\includegraphics[width=\textwidth]{FigureS2_heatmap_psiSite.png}
	\caption{
	    \textbf{Tissue-specific correlation structure for $\theta$.}
	    \textbf{(a-c)} Splice-site-centered and logit-transformed $\theta$
of the 10,000 most variable splice sites clustered by samples (columns and rows)
for three representative GTEx tissues: (a) suprapubic skin, (b)
left ventricle heart, and (c) whole blood. Red and blue relative depict high and 
low intron usage, respectively. Colored horizontal tracks display sequencing
center, batch, RNA integrity number (RIN), gender, age, and cause of death 
(DTHHRDY, Hardy scale classification) of the samples. \textbf{(d)} Boxplot of 
absolute values of between-sample correlations of row-centered logit-transformed
$\theta$ for 48 GTEx tissues before (orange) and after (green) correction 
for the latent space. The splice-site-centered $\theta$ values were 
clipped to the $[0.01, 0.99]$ interval before logit-transformation.
}
\end{figure}
\pagebreak


%% Figure S4
\begin{figure}[h]
\centering
	\includegraphics[width=\textwidth]{FigureS15_correlations_minN.png}
	\caption{
	    \textbf{Effect of total expression on sample correlation structure.}
Boxplots of absolute values of pairwise between-sample correlations of 
intron-centered logit-transformed splicing metrics (rows) for three representative GTEx tissues (columns)
before (orange) and after (green) correction for the latent space. The x-axis 
indicates the minimal total coverage (N) for an intron in both samples to be 
considered in the calculation of the pairwise sample correlation. The 
splicing metrics were clipped to the $[0.01, 0.99]$ interval before 
logit-transformation.
}
\end{figure}
\pagebreak


%% Figure S5
\begin{figure}[h]
	\centering
	\includegraphics[width=\textwidth]{FigureS4_finding_q_new.png}
	\caption{
	    \textbf{Finding the optimal latent space dimension $q$.}
	    \textbf{(a)} Area under the precision-recall curve  for recalling
artificially injected outliers (y-axis) against latent space dimension 
$q$ (x-axis) stratified by splicing metrics (rows) and three representative 
GTEx datasets (columns). Simulated outliers are generated using different 
scenarios: By shifting the splicing metrics away from its observed value 
(plain) or from its average across samples (dashed) and with shift of various 
amplitudes: 0.2 (green), 0.3 (orange), 0.5 (purple) or 0.7 (pink) as well as with 
amplitudes drawn uniformly in $[0.2,1]$ (black). For each scenario, 
the optimal latent space dimension $q$ is marked with a thicker dot.
\textbf{(b)} For each of the 48 GTEx tissues, 
the number of samples are plotted against the estimated latent
space dimension. The data is stratified by the splicing metrics (columns). 
The blue line represents a linear regression fit and the gray band around it
defines the 95\% confidence interval of the fit.
}
\end{figure}
\pagebreak


%% Figure S6
% \begin{figure}[h]
% \centering
% 	\includegraphics[width=\textwidth]{FigureS16_qEst_outlier_distr_psi5.png}
% 	\caption{
% 	    \textbf{Finding the optimal latent space dimension $q$ under different outlier injection procedures.}
% The area under the precision recall curve is plotted for a given latent space 
% dimension $q$. The plot is facetted by the GTEx tissue (columns) and the 
% distribution used to inject outliers (rows). The point size indicates whether the 
% tested dimesnion $q$ was found to be the optimal one. The linetype denotes whether 
% outlier counts were generated by shifting the observed sample $\psi_5$ or by shifting 
% the counts based on the mean $\psi_5$ of the intron. Colors indicate the minimal 
% $\Delta\psi_5$ by which outliers were introduced. 
% }
% \end{figure}
% \pagebreak


%% Figure S6 qq plots
\begin{figure}[h]
	\centering
	\includegraphics[width=\textwidth]{FigureS5_qq_plots.png}
	\caption{
	    \textbf{Quantile-quantile plots and the corresponding count ratios.}
	    \textbf{(a-l)} Quantile-quantile plots of randomly chosen introns based
on the $\psi_5$ metric. \textbf{(m-x)} Expression plots of the number of split
reads (K) over the total coverage (N) of the given donor site. The q-q plots (a-l)
correspond to the respective expression plot (m-x). The data is based on the 
suprapubic skin GTEx tissue.
}
\end{figure}
\pagebreak

%% Figure S7
\begin{figure}[h]
	\centering
	\includegraphics[width=\textwidth]{FigureS6_outlier_numbers.png}
	\caption{
	    \textbf{Distribution of extreme beta-binomial \textit{P} values across GTEx tissues.}
	    \textbf{(a)} The distribution of the number of events (y-axis) having a more 
	    significant \textit{P} value than a given cutoff (rows) is plotted
	    against the quantiles across samples within a tissue 
	    (x-axis, median and 90\%). The data is stratified by the different splicing 
	    metrics (columns, with green, orange, and purple for 
	    $\psi_5$, $\psi_3$, and $\theta$, respectively). 
	    Each distribution is based on the 48 GTEx tissues.
	    \textbf{(b)} Same as (a) but the frequency (y-axis) of \textit{P} values 
	    being smaller than a given cutoff (columns) across a tissue is plotted per 
	    splicing metrics (x-axis).
}
\end{figure}
\pagebreak


%% Figure S9
\begin{figure}[h]
\centering
	\includegraphics[width=\textwidth]{Figure4_precRec_clean_zscoreCheck/Skin_Not_Sun_Exposed_Suprapubic/uniformDistr/precRec_byJunctionGroup_psi5_clean.png}
	\caption{
	    \textbf{Splicing outlier detection benchmark in GTEx for $\psi_5$.}
        The proportion of simulated outliers among reported outliers (precision)
plotted against the proportion of reported simulated outliers among all 
simulated outliers (recall) for increasing \textit{P} values (FRASER, green; 
beta binomial, pink) or decreasing absolute PCA-based z scores (natural scale, 
orange, Supplementary Information; logit scale, purple). Moreover, all events 
with $|\Delta\psi_5| < 0.1$ are ranked last. The data are stratified by the mean 
coverage of the intron (columns) and by the injected absolute $\psi_5$ value (rows).
The cutoffs for each method are marked (FDR $<$ 0.1, circle; absolute z score $>$ 2, triangle).
The light ribbons around the curves indicate 95\% confidence bands estimated 
by bootstrapping.
}
\end{figure}
\pagebreak


%% Figure S10
\begin{figure}[h]
	\centering
	\includegraphics[width=\textwidth]{Figure4_precRec/Skin_Not_Sun_Exposed_Suprapubic/uniformDistr/precRec_byJunctionGroup_psi3.png}
	\caption{
	    \textbf{Splicing outlier detection benchmark in GTEx for $\psi_3$.}
	    The proportion of simulated outliers among reported outliers 
(precision, y-axis) plotted against the proportion of reported simulated 
outliers among all simulated outliers (recall, x-axis) for 
increasing beta-binomial \textit{P} values computed using count ratio 
expectations based on FRASER (green), a beta-binomial regression on the 
latent space (orange), or on raw count ratios (purple, na\"ive BB) 
and for decreasing absolute z scores on top of a beta-binomial regression
(pink) or PCA (light green). Additionally, all events with $|\Delta\psi| < 0.1$ are ranked last.
Plots are stratified equally by injected amplitudes ($\Delta\psi$, by row) 
and junction coverage (by column). The points indicate commonly 
applied cutoffs (FDR $< 0.1$ and $< 0.05$ and absolute z scores $>2$ and $>3$). 
Light ribbons around the curves depict 95\% confidence bands estimated by bootstrapping.
}
\label{fig:Bench3}
\end{figure}
\pagebreak


%% Figure S11
\begin{figure}[h]
	\centering
	\includegraphics[width=\textwidth]{Figure4_precRec/Skin_Not_Sun_Exposed_Suprapubic/uniformDistr/precRec_byJunctionGroup_psiSite.png}
	\caption{
	    \textbf{Splicing outlier detection benchmark in GTEx for $\theta$.}
	    Same as Figure \ref{fig:Bench3}, but based on the splicing efficiency metric $\theta$.
}
\end{figure}
\pagebreak


%% Figure S12
\begin{figure}[h]
\centering
	\includegraphics[width=\textwidth]{FigureS11_outlier_events_by_method.png}
	\caption{
	    \textbf{Distribution of aberrant splicing events in GTEx.}
	    Distribution of aberrant splicing events (y-axis) called using different 
	    cutoffs based on the FRASER normalization are plotted against
	    two quantiles (rows, 50\% and 90\%) across samples in a tissue for all 
	    48 GTEx tissues. The plot is stratified by the splicing metrics (rows) and 
	    commonly applied cutoffs (columns). Overall z score based cutoffs report 
	    in magnitudes more outlier than significant based and FDR controlled 
	    cutoff approaches.
}
\end{figure}
\pagebreak


%% Figure S13
\begin{figure}[h]
	\centering
	\includegraphics[width=\textwidth]{FigureS9_full_pr_plot.png}
	\caption{
	    \textbf{Performance benchmark using artificially injected outliers.} 
The proportion of simulated outliers among reported outliers (precision, y-axis)
plotted against the proportion of reported simulated outliers among all 
simulated outliers (recall, x-axis) for increasing beta-binomial \textit{P} values 
computed using count ratio expectations based on FRASER (green), 
a beta-binomial regression on the latent space (orange), or on raw count 
ratios (purple, na\"ive BB) and for decreasing absolute z scores on top of 
a beta-binomial regression (pink) or PCA (light green). The data is stratified 
by the different splice metrics: $\psi_5$, $\psi_3$, and $\theta$ (rows). The points indicate commonly 
applied cutoffs (FDR $< 0.1$ and $< 0.05$ and absolute z scores $>2$ and $>3$). 
Light ribbons around the curves depict 95\% confidence bands estimated by bootstrapping.
}
\end{figure}
\pagebreak


%% Figure S14
\begin{figure}[h]
\centering
	\includegraphics[width=\textwidth]{FigureS9_injectedVsFittedDpsi_PCA-BB-Decoder.png}
	\caption{
	    \textbf{Using a weighted beta-binomial loss for robustness against outlier data points.}
For each injected outlier data point, the effect size of the injection 
(y-axis, $\Delta\bullet$) is plotted against the predicted difference 
(x-axis, observed $\Delta\bullet -$ predicted $\Delta\bullet$) based on the beta-binomial 
regression fit. The plot is stratified by the three splicing metrics $\psi_5$,
$\psi_3$, and $\theta$ (rows) and by loss function used in the beta-binomial 
regression (columns, non weighted and weighted). The blue line corresponds to a 
linear regression. The red line indicates the diagonal, which would be the 
perfect fit. The data is based on the suprapubic skin GTEx tissue.
}
\end{figure}
\pagebreak


%% Figure S15
\begin{figure}[h]
\centering
	\includegraphics[width=\textwidth]{FigureS14_BenchmarkBySwapping.png}
	\caption{
	    \textbf{Recall analysis of injected outliers by interchanging read counts of alternatively spliced genes between tissues.}
	    Proportion of simulated outliers among reported outliers 
(precision, y-axis) against the proportion of reported simulated 
outliers among all simulated outliers (recall, x-axis) for 
increasing beta-binomial \textit{P} values computed using count ratio 
expectations based on FRASER (green) and on raw count ratios (orange, na\"ive BB) 
and Dirichlet-Multinomial \textit{P} values computed using the method from 
Kremer et al. (purple), LeafcutterMD (pink), and SPOT (light green). Light ribbons around the curves 
depict 95\% confidence bands estimated by bootstrapping.
}
\end{figure}
\pagebreak


%% Figure S16
\begin{figure}[h]
\centering
	\includegraphics[width=\textwidth]{FigureS10_GTEx_enrichment_pval.png}
	\caption{
	    \textbf{Gene-based rare variant enrichment analysis.}
	    \textbf{(a)} Enrichment using FRASER (y-axis) against enrichment using
different aberrant splicing detection methods (x-axis, columns) for rare variants
located in a splice region. The method applied are a beta-binomial regression on 
the latent space with (i) beta-binomial \textit{P} values and (ii) Gaussian
\textit{P} values, Dirichlet-Multinomial \textit{P} values based on
(iii) Kremer et al. and (iv) LeafcutterMD, (v) na\"ive beta-binomial \textit{P} values, (vi) 
PCA-based Gaussian \textit{P} values, and Dirichlet-Multinomial \textit{P}
values based on (vii) SPOT. The enrichment is calculated for different 
nominal \textit{P} value cutoffs (rows). Each dot represents a GTEx tissue. 
\textbf{(b)} The same as (a) but based on rare variants predicted to affect splicing by MMSplice.
\textbf{(c)} The same as (a) but an additional $|\Delta\psi| > 0.1$ cutoff was
applied on the aberrant splicing calls. \textbf{(d)} The same
as b but an additional $|\Delta\psi| > 0.1$ cutoff was applied on the aberrant 
splicing calls.
}
\end{figure}
\pagebreak


%% Figure S17
\begin{figure}[h]
\centering
	\includegraphics[width=\textwidth]{FigureS17_outlier_call_reproducibility.png}
	\caption{\textbf{Reproducibility of splicing outlier calls across GTEx tissues.}
        \textbf{(a)} Barplot of the number of gene-level events (y-axis) against
their reproducibility (x-axis) across GTEx tissues. The reproducibility is 
defined as the number of tissues an event is observered at a nominal \textit{P}
value $p < 10^{-3}$ given it was observerd at least once at a nominal \textit{P}
value $p < 10^{-5}$. The data is stratified by associated variant status and 
grouped by the different methods: FRASER (green), na\"ive beta-binomial (orange),
LeafcutterMD (purple), and SPOT (pink). 
\textbf{(b)} Same as (a) but plotted as the proportion (y-axis) of reproducible 
gene-level splicing outlier calls in GTEx tissues (number of tissues, x-axis). 
\textbf{(c)} Same as (b) but with at least one call at a nominal \textit{P} value $p < 10^{-7}$.
\textbf{(d)} Same as (b) but with at least one call at a nominal \textit{P} value $p < 10^{-9}$.
}
\end{figure}
\pagebreak


%% Figure S18
\begin{figure}[h]
\centering
	\includegraphics[width=\textwidth]{FigureS18_GTEx_enrichment_for_reproducible_events.png}
	\caption{
	    \textbf{Gene-based rare variant enrichment on reproducible splicing outlier calls.}
	    \textbf{(a)} Enrichment using FRASER (y-axis) against enrichment using
different aberrant splicing detection methods (x-axis, columns) for rare variants
located in a splice region. The methods applied are (i) na\"ive beta-binomial 
\textit{P} values, Dirichlet-Multinomial \textit{P} values based on 
(ii) LeafcutterMD and (iii) SPOT. The enrichment is calculated for different 
nominal \textit{P} value cutoffs (rows) and increased reproducibility cutoff: 
$p < 10^{-3}$ in at least 0, 1, or 9 other tissues (color).
Each dot represents a GTEx tissue. \textbf{(b)} The same as (a) but based on rare 
variants predicted to affect splicing by MMSplice.
\textbf{(c)} The same as (a) but an additional $|\Delta\psi| > 0.1$ cutoff was
applied on the aberrant splicing calls. \textbf{(d)} The same
as b but an additional $|\Delta\psi| > 0.1$ cutoff was applied on the aberrant 
splicing calls.
}
\end{figure}
\pagebreak

%% Figure S19
\begin{figure}[h]
\centering
	\includegraphics[width=\textwidth]{FigureS11_IntronRetention.png}
	\caption{
	    \textbf{Contribution of intron retention in aberrant splicing.}
\textbf{(a)} Barplot of the percentage of aberrant splicing events on the gene level
grouped by the detection metrics: alternative splicing only (purple, $\psi$ metric), 
intron retention (green, $\theta$ metric), and both (red). \textbf{(b)} Same as (a)
but with the absolute number of detected events.
}
\end{figure}
\pagebreak


%% Figure S19
\begin{figure}[h]
\centering
	\includegraphics[width=\textwidth]{FigureS14_power_analysis.png}
	\caption{
	    \textbf{Sample size analysis in Kremer et al.}
\textbf{(a)} The percentage of the recovered known disease-causing splicing 
outliers in Kremer et al. (y-axis) is plotted against the used sample size (x-axis). 
\textbf{(b)} The median of splicing outliers across all samples (y-axis) is plotted
against the used sample size (x-axis).
\textbf{(c)} The negative $\log_{10}$ \textit{P}-value for all known disease-causing
splicing outliers (y-axis) is plotted against the used sample size (x-axis). 
The color depicts the gene with a known splice defect.
}
\end{figure}
\pagebreak


\FloatBarrier


\section{Supplementary Information}

\subsection{Natural Z Score Calculation}

The natural z scores $z^{n}_{ij}$ are calculated per intron on the difference 
between the measured $\psi$ value including pseudocounts and the proportion 
expectation $\mu_{ij}$, as follows:

\begin{align*}
    z^n_{ij} =& \frac{\Delta\psi_{ij} - \overline{\Delta\psi_{j}}}{\text{sd}(\Delta\psi_{j})} \\
    \Delta\psi_{ij} =& \psi_{ij} - \mu_{ij} \\
                    =& \frac{k_{ij} + 1}{n_{ij} + 2} - \mu_{ij}. \\
\end{align*}


\subsection{Fitting of the parameters}

All notations are introduced in the Materials and Methods section.

\subsubsection*{Beta-binomial model}

We use the following parameterization of the beta-binomial distribution:

\begin{align*}
P(k| n, \alpha, \beta) = \frac{\Gamma(n+1)}{\Gamma(k+1)\Gamma(n-k+1)} 
\frac{\Gamma(\alpha + k)\Gamma(\beta + n - k)}{\Gamma(\alpha + \beta + n)}
\frac{\Gamma(\alpha + \beta)}{\Gamma(\alpha)\Gamma(\beta)},
\end{align*}

where 

\begin{align*}
\alpha &= \mu \left(\frac{1-\rho}{\rho}\right) \text{and } \beta = (\mu - 1)\left(\frac{\rho - 1}{\rho}\right).
\end{align*}

The variance of a beta-binomially distributed variable $X$ is given by:

\begin{align*}
\text{Var}(X) = n \mu (1 - \mu) (1 + (n-1)\rho )
\end{align*}

and the variance of the count ratio $\frac{k}{n}$ is then

\begin{align*}
\text{Var}\left(\frac{X}{n}\right) = \frac{\mu (1 - \mu) (1 + (n-1)\rho )}{n}.
\end{align*}

\subsubsection*{The negative log-likelihood of the beta-binomial distribution}

The negative log-likelihood ($\text{nll}$) of the model is given by:

\begin{align*}
\text{nll}=& - \sum_{ij}\log( \Gamma(n_{ij}+1)) + \sum_{ij}log(\Gamma(k_{ij}+1)) + \sum_{ij}\log(\Gamma(n_{ij}-k_{ij}+1)) \\
&- \sum_{ij}\log(\Gamma(\alpha_{ij}+k_{ij})) - \sum_{ij}\log(\Gamma(\beta_{ij}+n_{ij}-k_{ij})) + \sum_{ij}\log(\Gamma(\alpha_{ij} + \beta_{ij} + n_{ij})) \\
&+ \sum_{ij}\log(\Gamma(\alpha_{ij})) + \sum_{ij}\log(\Gamma(\beta_{ij})) - \sum_{ij}\log(\Gamma(\alpha_{ij} + \beta_{ij})).
\end{align*}

\subsubsection*{Fitting the intra-class correlation parameter $\rho$}

For the optimization of the intra-class correlation parameter $\rho$ only the 
$\rho$ depending terms in $\text{nll}$ need to be considered and yield the 
following truncated form of the negative log likelihood, with pseudocounts of $1$
and $2$ added to $k$ and $n$, respectively:

\begin{align*}
\text{nll}_{\rho} =& 
     - \sum_{ij}\log(\Gamma(\alpha_{ij}+k_{ij})) % a + k
     - \sum_{ij}\log(\Gamma(\beta_{ij}+n_{ij}-k_{ij})) % b + n - k
     + \sum_{ij}\log(\Gamma(\alpha_{ij} + \beta_{ij} + n_{ij})) \\ % a + b + n
    &+ \sum_{ij}\log(\Gamma(\alpha_{ij})) % a
     + \sum_{ij}\log(\Gamma(\beta_{ij})) % b
     - \sum_{ij}\log(\Gamma(\alpha_{ij} + \beta_{ij})). % a + b
\end{align*}

In the following $y_{ij}$ is an element of $\matr{Y}$ defined as:

\begin{align}\label{eq:Y}
\matr{Y} &= \matr{\widetilde{X}} \matr{W}_e \matr{W}_d + \matr{b},
\end{align}

where the element $\widetilde{x}_{ij}$ of the matrix $\matr{\widetilde{X}}$ is given by: 

\begin{align*}
\widetilde{x}_{ij} &= x_{ij} - \bar{x}_j, \\
x_{ij} &= \text{logit} \left (\frac{k_{ij}+1}{n_{ij}+2} \right), \\
\text{logit}(a) &= \log \frac{a}{1-a}.
\end{align*}

The expectations $\mu_{ij}$ are then modelled by:

\begin{align*}
\mu_{ij} &= \sigma \left (y_{ij} \right) = \frac{e^{y_{ij}}}{1 + e^{y_{ij}}}
\end{align*}


Hence, $\text{nll}_{\rho}$ can be rewritten as:

\begin{align*}
\text{nll}_{\rho} =& 
     - \sum_{ij}\log\left(\Gamma\left(\frac{ e^{y_{ij}} }{ 1+e^{y_{ij}} }\frac{ 1-\rho_{ij} }{ \rho_{ij} } + k_{ij} + 1 \right) \right) \\ % a + k
    &- \sum_{ij}\log\left(\Gamma\left(\left(\frac{ e^{y_{ij}} }{ 1+e^{y_{ij}} } - 1 \right) \frac{ \rho_{ij} - 1 }{ \rho_{ij} } + n_{ij} - k_{ij} + 1\right) \right) \\ % b + n - k
    &+ \sum_{ij}\log\left(\Gamma\left(\frac{ 1 - \rho_{ij} }{ \rho_{ij} } + n_{ij} + 2\right) \right) \\ % a + b + n
    &+ \sum_{ij}\log\left(\Gamma\left(\frac{ e^{y_{ij}} }{ 1+e^{y_{ij}} }\frac{ 1-\rho_{ij} }{ \rho_{ij} }\right) \right) \\ % a
    &+ \sum_{ij}\log\left(\Gamma\left( \left (\frac{ e^{y_{ij}} }{ 1+e^{y_{ij}} } - 1 \right) \frac{ \rho_{ij} - 1 }{ \rho_{ij} }\right) \right) \\ % b
    &- \sum_{ij}\log\left(\Gamma\left(\frac{ 1 - \rho_{ij} }{ \rho_{ij} } \right) \right) % a + b
\end{align*}

To fit $\rho$, we use $optimize$ provided in the R base package \textit{stats}.
To not run into convergence issues or numerical instability of the log gamma function, 
we estimate the value of the log gamma function if not $-35 < y_{ij} < 30$.

\subsection{Alternative approaches to model the encoder and decoder}

For the optimization of the autoencoder model only the terms of $\text{nll}$
that are dependent on $\matr{W}_e$ or $\matr{W}_d$ need to be considered. 
Since
\begin{align*}
\alpha + \beta = \frac{1-\rho}{\rho}
\end{align*}
is independent of $\mu$ and therefore independent of $\matr{W}_e$ and $\matr{W}_d$,
we do not have to consider the terms containing $\alpha + \beta$ and yield the 
following truncated form of the negative log likelihood, with pseudocounts of 
$1$ and $2$ added to $k$ and $n$, respectively:

\begin{align} \label{eq:nll-w}
\text{nll}_{\matr{W}} =& \sum_{ij}\log(\Gamma(\alpha_{ij})) + \sum_{ij}\log(\Gamma(\beta_{ij})) \\
&- \sum_{ij}\log(\Gamma(\alpha_{ij}+k_{ij}+1)) - \sum_{ij}\log(\Gamma(\beta_{ij}+n_{ij}-k_{ij}+1)).
\end{align}

Hence,  $\text{nll}_{\matr{W}}$ can be rewritten as:

\begin{align*}
\text{nll}_{\matr{W}} =& \sum_{ij}\log\left(\Gamma\left(\frac{ e^{y_{ij}} }{ 1+e^{y_{ij}} }\frac{ 1-\rho_{ij} }{ \rho_{ij} }\right) \right) \\
&+ \sum_{ij}\log\left(\Gamma\left( \left (\frac{ e^{y_{ij}} }{ 1+e^{y_{ij}} } - 1 \right) \frac{ \rho_{ij} - 1 }{ \rho_{ij} }\right) \right) \\
&- \sum_{ij}\log\left(\Gamma\left(\frac{ e^{y_{ij}} }{ 1+e^{y_{ij}} }\frac{ 1-\rho_{ij} }{ \rho_{ij} } + k_{ij} + 1 \right) \right) \\ 
&- \sum_{ij}\log\left(\Gamma\left( \left (\frac{ e^{y_{ij}} }{ 1+e^{y_{ij}} } - 1 \right) \frac{ \rho_{ij} - 1 }{ \rho_{ij} } + n_{ij} - k_{ij} + 1\right) \right) 
\end{align*}


We use L-BFGS\cite{byrd1995} as implemented in $optim$  to fit the autoencoder
model as described in Methods.


\subsubsection*{Update of the encoder and decoder matrix}

The updating of the matrix $\matr{W}_d$ is performed intron-wise whereas the 
encoder matrix $\matr{W}_e$ is performed on the full matrix. For each update step,
the intron-wise or matrix-wise average negative log likelihood is minimized. 
To not run into convergence issues or numerical instability of the digamma function, 
we estimate the value of the digamma function $\psi$ if not $-35 < y_{ij} < 30$.
From Equation  \ref{eq:Y} and Equation \ref{eq:nll-w}, we obtain the gradients:

\begin{align*}
    \frac{d\text{nll}}{d\matr{W}_e} &= \matr{\widetilde{X}}^T \matr{A} \matr{W}_d + \matr{\widetilde{X}}^T \matr{B} \matr{W}_d - \matr{\widetilde{X}}^T \matr{C} \matr{W}_d - \matr{\widetilde{X}}^T \matr{D} \matr{W}_d  \\  
    \frac{d\text{nll}}{d\matr{W}_d} &= \matr{A}^T \matr{\widetilde{X}} \matr{W}_e + \matr{B}^T \matr{\widetilde{X}}\matr{W}_e - \matr{C}^T \matr{\widetilde{X}}\matr{W}_e - \matr{D}^T \matr{\widetilde{X}} \matr{W}_e \\
    \frac{d\text{nll}}{db_j} &= \sum_{i} a_{ij} + b_{ij} - c_{ij} - d_{ij}
    \end{align*},

where the components of the matrices $\matr{A}, \matr{B}, \matr{C}$ and $\matr{D}$ are computed by:
\begin{align*}
	a_{ij} &= \psi \left( \frac{ e^{y_{ij}} }{ 1+e^{y_{ij}} } \cdot r_{ij} \right) \cdot r_{ij} \cdot v_{ij}    \\
	b_{ij} &= \psi \left( \left (\frac{ e^{y_{ij}} }{ 1+e^{y_{ij}} } - 1 \right) \cdot (-r_{ij} \right) \cdot (-r_{ij}) \cdot v_{ij} \\
	c_{ij} &= \psi \left( \frac{ e^{y_{ij}} }{ 1+e^{y_{ij}} } \cdot r_{ij} + k_{ij} + 1 \right) \cdot r_{ij} \cdot v_{ij}    \\
	d_{ij} &= \psi \left( \left (\frac{ e^{y_{ij}} }{ 1+e^{y_{ij}} } - 1 \right) \cdot (-r_{ij} + n_{ij} - k_{ij} + 1 \right) \cdot (-r_{ij}) \cdot v_{ij} \\
	v_{ij} &= \frac{e^{y_{ij}}}{\left(1+e^{y_{ij}}\right)^2} \\
    r_{ij} &= \frac{1 - \rho_{j} }{ \rho_j }  
\end{align*}.





\bibliography{bibliography}

\end{document}
